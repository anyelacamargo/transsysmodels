\documentclass[a4paper,fleqn]{article}

\usepackage[T1]{fontenc}
\usepackage{mathptmx}
\usepackage[english]{babel}
\usepackage{alltt}
\usepackage{moreverb}
\usepackage{pstricks}


\textheight 23cm \textwidth 16cm
\oddsidemargin 0pt \evensidemargin 0pt

\newcommand{\todo}[1]{
  \rule{0pt}{0pt}\marginpar{{\color{blue}\rule{1ex}{1ex}}}
  {[\textbf{\color{blue}todo:} #1]}}


\begin{document}
\title{Notes on Integrative transsys Modelling of Wheat at
  GRN and Phenomics Levels}
\author{Anyela Camargo, Jan T Kim}
\maketitle


\section{Objectives}

\subsection{Key Objectives}

\subsubsection{Demonstrate Pixel Frequency Based Scoring}

Wheat plants have been imaged (and scored in terms of developmental
stage) at multiple time points. The objectives are
\begin{itemize}
\item to generate comparable images from L-transsys simulations,
\item to devise a score to compare images
\item to demonstrate the ability / potential of this score to
  discriminate between models.
\end{itemize}


\subsection{Perspectives}

\subsubsection{Integrate Gene Expression Level}

\todo{summarise idea and problems with Gregersen stuff}

\subsubsection{Ideas, Open Issues}

\begin{itemize}
\item Consider a score that does not require quantisation and that is
  less sensitive and degrades gracefully in the presence of random
  fluctuations?
\item Capture and model biological variation
\end{itemize}


\section{Work}

\subsection{Key Objectives}

\subsubsection{Pixel Frequency Based Scoring}

From each image, a colour frequency profile comprised of the frequency
values of pixels of each colour is determined. These colour frequency
profiles are then compared using a score based on Euclidean or
correlation distance.

Colours will be coarsely quantised (at least in initial approaches),
to avoid excessive sensitivity to colour settings in simulations
and to random fluctuation in empirical measurements (e.g.\ due to
changes in ambient light conditions).


\textbf{Requirements and Progress:}

\begin{itemize}
\item \textit{Generate images from L-transsys model:} Prototype
implemented
\item \textit{Compute pixel frequencies from simulations:} Prototype
  implemented, may need reviewing / updating.
\item \textit{Pre-process empirical images:} Done (partially?)
\item \textit{Devise pixel frequency profile data structure suitable
    for empirical and simulated data:} Prototype was working but is
  now defunct due to changes. Review ``\texttt{cdata}'' for
  suitability.
\end{itemize}


\appendix

\section{Models}

\subsection{\texttt{wheat105.trl}}

\begin{footnotesize}
\verbatimtabinput{wheat105.trl}
\end{footnotesize}


\section{Analysis Code}

\subsection{\texttt{senescence\_imageanalysis.R}}

\begin{footnotesize}
\verbatimtabinput{senescence_imageanalysis.R}
\end{footnotesize}


\end{document}


%%% Local Variables: 
%%% mode: latex
%%% TeX-master: t
%%% End: 
