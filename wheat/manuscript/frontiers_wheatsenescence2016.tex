%%%%%%%%%%%%%%%%%%%%%%%%%%%%%%%%%%%%%%%%%%%%%%%%%%%%%%%%%%%%%%%%%%%%%%%%%%%%%%%%%%%%%%%%%%%%%%%%%%%%%%%%%%%%%%%%%%%%%%%%%%%%%%%%%%%%%%%%%%%%%%%%%%%%%%%%%%%
% This is just an example/guide for you to refer to when submitting manuscripts to Frontiers, it is not mandatory to use Frontiers .cls files nor frontiers.tex  %
% This will only generate the Manuscript, the final article will be typeset by Frontiers after acceptance.                                                 %
%                                                                                                                                                         %
% When submitting your files, remember to upload this *tex file, the pdf generated with it, the *bib file (if bibliography is not within the *tex) and all the figures.
%%%%%%%%%%%%%%%%%%%%%%%%%%%%%%%%%%%%%%%%%%%%%%%%%%%%%%%%%%%%%%%%%%%%%%%%%%%%%%%%%%%%%%%%%%%%%%%%%%%%%%%%%%%%%%%%%%%%%%%%%%%%%%%%%%%%%%%%%%%%%%%%%%%%%%%%%%%

%%% Version 3.1 Generated 2015/22/05 %%%
%%% You will need to have the following packages installed: datetime, fmtcount, etoolbox, fcprefix, which are normally inlcuded in WinEdt. %%%
%%% In http://www.ctan.org/ you can find the packages and how to install them, if necessary. %%%

\documentclass{frontiersSCNS} % for Science, Engineering and Humanities and Social Sciences articles
%\documentclass{frontiersHLTH} % for Health articles
%\documentclass{frontiersFPHY} % for Physics and Applied Mathematics and Statistics articles

%\setcitestyle{square}
\usepackage{url,hyperref,lineno,microtype}
\usepackage[onehalfspacing]{setspace}
\linenumbers


% Leave a blank line between paragraphs instead of using \\


\def\keyFont{\fontsize{8}{11}\helveticabold }
\def\firstAuthorLast{tbd {et~al.}} %use et al only if is more than 1 author
\def\Authors{Anyela V.\ Camargo Rodr\'{i}guez\,$^{1,*}$, Jan T.\ Kim\,$^{2}$ and tbd\,$^3$}
% Affiliations should be keyed to the author's name with superscript numbers and be listed as follows: Laboratory, Institute, Department, Organization, City, State abbreviation (USA, Canada, Australia), and Country (without detailed address information such as city zip codes or street names).
% If one of the authors has a change of address, list the new address below the correspondence details using a superscript symbol and use the same symbol to indicate the author in the author list.
\def\Address{
%  $^{1}$Laboratory X, Institute X, Department X, Organization X, City X , State XX (only USA, Canada and Australia), Country X \\
  $^{1}$ Institute of Biological, Environmental and Rural Sciences,
  Aberystwyth University, Aberystwyth, United Kingdom \\
  $^{2}$The Pirbright Institute, Pirbright, United Kingdom \\
  $^{3}$ (should we also include Alison Bentley?
}
% The Corresponding Author should be marked with an asterisk
% Provide the exact contact address (this time including street name and city zip code) and email of the corresponding author
\def\corrAuthor{Corresponding Authors}
\def\corrAddress{Institute of Biological, Environmental and Rural Sciences,
  Aberystwyth University, Aberystwyth, United Kingdom}
\def\corrEmail{avc1@aber.ac.uk}


\newcommand{\todo}[1]{
  \rule{0pt}{0pt}\marginpar{{\color{blue}\rule{1ex}{1ex}}}
  {[\textbf{\color{blue}todo:} #1]}}


\begin{document}
\onecolumn
\firstpage{1}

\title[Integrative Wheat Senescence Modelling]{Towards Integrative
  Modelling of Gene Expression and Phenomics in Wheat Senescence}

\author[\firstAuthorLast ]{\Authors} %This field will be automatically populated
\address{} %This field will be automatically populated
\correspondance{} %This field will be automatically populated

\extraAuth{}% If there are more than 1 corresponding author, comment this line and uncomment the next one.
%\extraAuth{corresponding Author2 \\ Laboratory X2, Institute X2, Department X2, Organization X2, Street X2, City X2 , State XX2 (only USA, Canada and Australia), Zip Code2, X2 Country X2, email2@uni2.edu}


\maketitle

%%%%%%%%%%%%%%%%%%%%%%%%%%%%%%%%%%%%%%%%%%%%%%%%%%%%%%%%%%%%%%%%%%%%%%%%%%%%%%%%%%%%%%%%%%%%%%%%%%%%%%%%%%%%%%%%%%%%%%%%%%%%%%%%%%%%%%%%%%%%%%%%%%%%%%%%%%%%%%%%%%%%%%%%%%%%%%%%%%%%%%%%%%%%%%%%%%%%%%%%%%%%%%%%%%%%%%%%%%%%%%%%%%%%%%%
%%% The sections below are for reference only.
%%%
%%% For Original Research Articles, Clinical Trial Articles, and Technology Reports the section headings should be those appropriate for your field and the research itself. It is recommended to organize your manuscript in the
%%% following sections or their equivalents for your field:
%%% Abstract, Introduction, Material and Methods, Results, and Discussion.
%%% Please note that the Material and Methods section can be placed in any of the following ways: before Results, before Discussion or after Discussion.
%%%
%%%For information about Clinical Trial Registration, please go to http://www.frontiersin.org/about/AuthorGuidelines#ClinicalTrialRegistration
%%%
%%% For Clinical Case Studies the following sections are mandatory: Abstract, Introduction, Background, Discussion, and Concluding Remarks.
%%%
%%% For all other article types there are no mandatory sections.
%%%%%%%%%%%%%%%%%%%%%%%%%%%%%%%%%%%%%%%%%%%%%%%%%%%%%%%%%%%%%%%%%%%%%%%%%%%%%%%%%%%%%%%%%%%%%%%%%%%%%%%%%%%%%%%%%%%%%%%%%%%%%%%%%%%%%%%%%%%%%%%%%%%%%%%%%%%%%%%%%%%%%%%%%%%%%%%%%%%%%%%%%%%%%%%%%%%%%%%%%%%%%%%%%%%%%%%%%%%%%%%%%%%%%%%

\begin{abstract}

%%% Leave the Abstract empty if your article falls under any of the following categories: Editorial Book Review, Commentary, Field Grand Challenge, Opinion or specialty Grand Challenge.
\section{}
%As a primary goal, the abstract should render the general significance and conceptual advance of the work clearly accessible to a broad readership. References should not be cited in the abstract.
% For full guidelines regarding your manuscript please refer to \href{http://www.frontiersin.org/about/AuthorGuidelines}{Author Guidelines} \\ or \textbf{Table \ref{Tab:01}} for a summary according to article type.

Senescence in wheat (and other annual crops) maximises resources
directed to seeds and fruits, and therefore it is key to maximising
yield. Senescence is a complex process involving multiple levels of
organisation, with the levels of gene expression, morphogenesis, and
phenotypic re-organisation (prominently resulting in yellowing)
playing key roles.

We present a computational multiscale modelling approach that
explicitly represents regulatory networks and morphological
phenotypes, and which is capable of integrating empirical
transcriptomic and phenomic data, and give an initial demonstration of
assessing our multiscale models based on multidimensional times series
data. These data were generated from images captured by a phenomics
platform dring the development of an individual plant. 

The modelling assessment approach is designed to facilitate
integration of a very broad range of phenomics and transcriptomics
data. We plan to incorporate gene expression data and to extend the
model to include environmental measurements (such as temperature and
light conditions), as well as further types of phenomics data. The
long term perspective of our work is to apply computational modelling
to advance principled understanding of senescence, and thereby
enabling its prediction in changing or unprecedented environmental
conditions, with a view to harnessing it as a tool for improving
resilience to climate change.

\tiny
 \keyFont{ \section{Keywords:} Wheat, Senescence, Computational
   Modelling, Phenomics, Transcriptomics } %All article types: you may provide up to 8 keywords; at least 5 are mandatory.
\end{abstract}

\section{Introduction}

% For Original Research Articles, Clinical Trial Articles, and Technology Reports the introduction should be succinct, with no subheadings.
%
% For Clinical Case Studies the Introduction should include symptoms at presentation, physical exams and lab results.
%
Senescence generally describes the final developmental stage of a
plant organ (or structure, possibly encompassing the entire plant)
\todo{find suitable recent senescence review to cite here}. The
physiological and morphological changes during senescence result from
dismantling structures that are no longer used or required, associated
with the transfer of resources (such as metabolites and nutrients) to
other structures. In annual crops, these prioritised structures
typically are fruits and seeds, i.e.\ those that make up the yield of
the crop.

Senescence is a key decision point in a plant's life cycle. On the one
heand, premature senescence of structures of photosynthesis,
especially the flag leaf in wheat, result in limitations to assimilate
production and yield. On the other hand, delaying senescence may
result in reduced yield by unnecessarily holding resources in the flag
leaf. Even more importantly, senescence interacts with other
developmental processes and specifically it may promote maturation of
the ear and the grains, so a delay in senescence may detract from
yield by interfering with maturation. \todo{need some more background
  here --- I have some recollection that senescence and maturation
  mutually promote each other, but can't come up with a citation.}
Thus, the onset of senescence is an important adaptive feature which
is exquisitely controlled. On the individual level, plants use inputs
sensed from the environment to trigger senescence via processing by
regulatory networks. On the evolutionary level, these mechanisms
themselves are subject to selection.


%\begin{methods}
\section{Material \& Methods}

\subsection{Computational Modelling}

\todo{Outline transsys and describe modelling approach} \cite{Kim2000},
\cite{Camargo2011_simgenex}, \cite{Camargo2012_dogenets}.


\subsection{Data}

\todo{Anyela -- can you write a description of the phenomics data and
  its processing into real-valued time series per colour here?}


\subsection{Model Scoring}

\todo{Describe distance between pixel colour frequency profiles and
  briefly outline properties / justify starting with this simple
  measure}


% \begin{table}[!t]
% \textbf{\refstepcounter{table}\label{Tab:01} Table \arabic{table}.}{ Maximum size of the Manuscript }

% \processtable{ }
% {\begin{tabular}{lllll}\toprule
%  & Abstract max. legth (incl. spaces) & Figures or tables & Manuscript max. length \\\midrule
% Clinical Case Study & & & &\\
% Clinical Trial & & & &\\
% Hypothesis and Theory & & & &\\
% Methods & 2000 characters  & 15 & 12000 words \\
% Original Research & & & &\\
% Review & & & &\\
% Technology Report & & & &\\\midrule
% Focused Review & 2000 characters & 5 & 5000 words \\\midrule
% CPC &  1250 characters& 6 & 2500 words  \\\midrule
% Perspective & 1250 characters & 2 & 3000 words  \\
% Mini Review & & & &\\\midrule
% Data Report & None & 2 & 3000 words\\\midrule
% Classification & 1250 characters & 10 & 2000 words \\\midrule
% Editorial & None & None & 1000 words  \\\midrule
% Frontiers Commentary  & & &\\
% General Commentary & None & 1 & 1000 words\\
% Book review & & & \\\midrule
% Opinion   & & &\\
% Specialty Grand Challenge & None & 1 & 2000 words\\
% Field Grand Challenge & & & &\\\botrule
% \end{tabular}}{}
% \end{table}

% Please note that large tables covering several pages cannot be
% included in the final PDF for formatting reasons. These tables will be
% published as supplementary material on the online article abstract
% page at the time of acceptance. The author will notified during the
% typesetting of the final article if this is the case. A link in the
% final PDF will direct to the online material.

\section{Results}

\todo{Show the three variants and their scores}

% \subsection{Figures}

% Frontiers requires figures to be submitted individually, in the same
% order as they are referred to in the manuscript. Figures will then
% be automatically embedded at the bottom of the submitted manuscript.
% Kindly ensure that each table and figure is mentioned in the text
% and in numerical order. Permission must be obtained for use of
% copyrighted material from other sources (including the web). Please
% note that it is compulsory to follow figure instructions. Figures
% which are not according to the guidelines will cause substantial
% delay during the production process.


% \begin{table}[!t]
% \textbf{\refstepcounter{table}\label{Tab:02} Table \arabic{table}.}{ Resolution Requirements for the figures}

% \processtable{}
% {\begin{tabular}{lllll}\toprule
% Image Type & Description & Format & Color Mode & Resolution\\\midrule
% Line Art & An image composed of lines and text,  & TIFF, JPEG & RGB, Bitmap & 900 - 1200 dpi\\
%            & which does not contain tonal or shaded areas.& & &\\
%            Halftone & A continuous tone photograph, which contains no text. & TIFF, EPS, JPEG & RGB, Grayscale & 300 dpi\\
% Combination & Image contains halftone + text or line art elements. & TIFF, JPEG & RGB,Grayscale & 600 - 900 dpi\\\botrule
% \end{tabular}}{}
% \end{table}

% \textbf{Table \ref{Tab:02}} shows the resolution requirements for the figures. The figures must be legible:
% \begin{enumerate}
% \item The smallest visible text is no less than 8 points in height, when viewed at actual size.
% \item Solid lines are not broken up.
% \item Image areas are not pixelated or stair stepped.
% \item Text is legible and of high quality.
% \item Any lines in the graphic are no smaller than 2 points width.
% \item The actual size of the figure must be of at least 8.5 cm.
% \end{enumerate}

% \subsection{Nomenclature}
% \begin{itemize}
% \item The use of abbreviations should be kept to a minimum. Non-standard abbreviations should be avoided unless they appear at least four times, and defined upon first use in the main text. Consider also giving a list of non-standard abbreviations at the end, immediately before the Acknowledgments.
% \item Gene symbols should be italicized; protein products are not italicized.
% \item Chemical compounds and biomolecules should be referred to using systematic nomenclature, preferably using the recommendations by IUPAC.
% \item We encourage the use of Standard International Units in all manuscripts.
% \item To take part in the Resource Identification Initiative, please cite antibodies, genetically modified organisms, software tools, data, databases and services using the corresponding catalog number and RRID in your current manuscript. For more information about the project and for steps on how to search for an RRID, please click \href{http://www.frontiersin.org/files/pdf/letter_to_author.pdf}{here}.
% \end{itemize}

% \begin{equation}
% \sum x+ y =Z\label{eq:01}
% \end{equation}

\section{Discussion}

The work presented here introduces a technique for building multiscale
computational models that integratively capture gene regulatory
networks and phenotypic properties at the levels of organs and
thewhole plant. Integration of empirical phenomic data requires
developing a method to simulate the same type of data based on the
model, and a distance measure to quantitatively score the deviation of
a simulated data set from the empirical data.

\todo{Discuss options for feeding environmental signals into
  simulation, including limitations resulting from using L-systems,
  approaches to overcome these, and alternatives}

% Additional Requirements:
% \subsection{Corrections}

% If you need to communicate important changes to a published article please submit a General Commentary. Submit the article with the title “Corrigendum: Original Title of Article”.

% \subsection{Commentaries on Articles}

% At the beginning of your Commentary, please provide the citation of the article commented on. Rebuttals may be submitted in response to Commentaries; our limit in place is one commentary and one response. Rebuttals should also be submitted as General Commentary articles.

% \subsection{Human Search and Animal Research}

% All experiments on live vertebrates or higher invertebrates must be performed in accordance with relevant institutional and national guidelines and regulations. In the manuscript, authors must identify the committee approving the experiments and must confirm that all experiments conform to the relevant regulatory standards. For manuscripts reporting experiments on human subjects, authors must identify the committee approving the experiments and must also include a statement confirming that informed consent was obtained from all subjects. In Original Research Articles and Clinical Trial Articles these statements should appear in the Materials and Methods section.

% \subsection{Clinical Trial Registration}

% Clinical trials should be registered in a public trials registry in order to become the object of a publication at Frontiers. Trials must be registered at or before the start of patient enrollment. A clinical trial is defined as "any research study that prospectively assigns human participants or groups of humans to one or more health-related interventions to evaluate the effects on health outcomes."(\href{www.who.int/ictrp/en}{www.who.int/ictrp/en}). A list of acceptable registries can be found at \href{www.who.int/ictrp/en}{www.who.int/ictrp/en} and \href{www.icmje.org}{www.icmje.org}.

% \subsection{Inclusion of Proteomics Data}

% Authors should provide relevant information relating to how the peptide/protein matches were undertaken, including methods used to process and analyze data, false discovery rates (FDR) for large-scale studies and threshold or cut-off rates for peptide and protein matches. Further information could include software used, mass spectrometer type, sequence database and version, number of sequences in database, processing methods, mass tolerances used for matching, variable/fixed modifications, allowable missed cleavages, etc.

% Authors should provide as supplementary material information used to identify proteins and/or peptides. This should include information such as accession numbers, observed mass (m/z), charge, delta mass, matched mass, peptide/protein scores, peptide modification, miscleavages, peptide sequence, match rank, matched species (for cross species matching), number of peptide matches, ambiguous protein/peptide matches should be indicated, etc.
% For quantitative proteomics analyses authors should provide information to justify the statistical significance including biological replicates, statistical methods, estimates of uncertainty and the methods used for calculating error.

% For peptide matches with biologically relevant post-translational modifications (PTM) and for any protein match that has occurred using a single mass spectrum, authors should include this information as raw data, annotated spectra or submit data to an online repository (recommended option).
% Authors are encouraged to submit raw or matched data and 2-DE images to public proteomics repositories. Submission codes and/or links to data should be provided within the manuscript.

\subsection{Data Sharing}

\todo{Can you release the image data consistently with the Frontiers
  policy below?}

Frontiers supports the policy of data sharing, and authors are advised
to make freely available any materials and information described in
their article, and any data relevant to the article (while not
compromising confidentiality in the context of human-subject research)
that may be reasonably requested by others for the purpose of academic
and non-commercial research. In regards to deposition of data and data
sharing through databases, Frontiers urges authors to comply with the
current best practices within their discipline.

\section*{Disclosure/Conflict-of-Interest Statement}
%Frontiers follows the recommendations by the International Committee of Medical Journal Editors (http://www.icmje.org/ethical_4conflicts.html) which require that all financial, commercial or other relationships that might be perceived by the academic community as representing a potential conflict of interest must be disclosed. If no such relationship exists, authors will be asked to declare that the research was conducted in the absence of any commercial or financial relationships that could be construed as a potential conflict of interest. When disclosing the potential conflict of interest, the authors need to address the following points:
%•	Did you or your institution at any time receive payment or services from a third party for any aspect of the submitted work?
%•	Please declare financial relationships with entities that could be perceived to influence, or that give the appearance of potentially influencing, what you wrote in the submitted work.
%•	Please declare patents and copyrights, whether pending, issued, licensed and/or receiving royalties relevant to the work.
%•	Please state other relationships or activities that readers could perceive to have influenced, or that give the appearance of potentially influencing, what you wrote in the submitted work.

\todo{jtk: I don't have any conflicts to declare, others please
  confirm by adding their name to this todo}

The authors declare that the research was conducted in the absence of
any commercial or financial relationships that could be construed as a
potential conflict of interest.

\section*{Author Contributions}
%When determining authorship the following criteria should be observed:
%•	Substantial contributions to the conception or design of the work; or the acquisition, analysis, or interpretation of data for the work; AND
%•	Drafting the work or revising it critically for important intellectual content; AND
%•	Final approval of the version to be published ; AND
%•	Agreement to be accountable for all aspects of the work in ensuring that questions related to the accuracy or integrity of any part of the work are appropriately investigated and resolved.
%Contributors who meet fewer than all 4 of the above criteria for authorship should not be listed as authors, but they should be acknowledged. (http://www.icmje.org/roles_a.html)

% The statement about the authors and contributors can be up to several sentences long, describing the tasks of individual authors referred to by their initials and should be included at the end of the manuscript before the References section.


\section*{Acknowledgments}


\textit{Funding\textcolon} \todo{probably the grant codes of the
  LATPPN should go here --- Anyela?}

\section*{Supplemental Data}
Supplementary Material should be uploaded separately on submission, if there are Supplementary Figures, please include the caption in the same file as the figure. LaTeX Supplementary Material templates can be found in the Frontiers LaTeX folder



\bibliographystyle{frontiersinSCNS_ENG_HUMS} % for Science, Engineering and Humanities and Social Sciences articles, for Humanities and Social Sciences articles please include page numbers in the in-text citations
%\bibliographystyle{frontiersinHLTH&FPHY} % for Health and Physics articles
\bibliography{bioinfo}

%%% Upload the *bib file along with the *tex file and PDF on submission if the bibliography is not in the main *tex file

\section*{Figures}

%%% Use this if adding the figures directly in the mansucript, if so, please remember to also upload the files when submitting your article
%%% There is no need for adding the file termination, as long as you indicate where the file is saved. In the examples below the files (logo1.jpg and logo2.eps) are in the Frontiers LaTeX folder
%%% If using *.tif files convert them to .jpg or .png

\begin{figure}[h!]
\begin{center}
\includegraphics[width=10cm]{logo1}% This is a *.jpg file
\end{center}
 \textbf{\refstepcounter{figure}\label{fig:01} Figure \arabic{figure}.}{ Enter the caption for your figure here.  Repeat as  necessary for each of your figures }
\end{figure}

%\begin{figure}
%\begin{center}
%\includegraphics[width=10cm]{logo2}% This is an *.eps file
%\end{center}
%\textbf{\refstepcounter{figure}\label{fig:02} Figure \arabic{figure}.}{ Enter the caption for your figure here.  Repeat as  necessary for each of your figures }
%\end{figure}

%%% If you don't add the figures in the LaTeX files, please upload them when submitting the article.

%%% Frontiers will add the figures at the end of the provisional pdf automatically %%%

%%% The use of LaTeX coding to draw Diagrams/Figures/Structures should be avoided. They should be external callouts including graphics.

\end{document}

% > I’d like to follow up on your contribution to the Research Topic on “Plant Phenotyping and Phenomics for Plant Breeding” hosted by Dr(s) Rodomiro Ortiz, José Luis Araus, Alejandro Del Pozo, Gustavo A. Lobos, John Doonan in Frontiers in Plant Science, section Crop Science and Horticulture.
% >
% > The manuscript submission deadline set to Feb 12, 2016 is approaching, and in case you anticipate any delay in submitting your manuscript, please inform both the Topic Editors and the Frontiers Editorial Office as soon as possible.
% >
% > I would also like to remind you that contributions can be of different article types (Original Research, Methods, Hypothesis & Theory, etc.). Please see the author guidelines below for the different article types:
% > http://frontiersin.org/Journal/AuthorInfo/ManuscriptGuidelines.aspx
% >
% > Manuscripts should be submitted directly online, by clicking on “Submit Manuscript”:
% > http://frontiersin.org/Crop_Science_and_Horticulture/researchtopics/Plant_Phenotyping_and_Phenomics_for_Plant_Breeding_1/4223
% >
% > For a list of article types and related publishing fees, please see here:
% > http://www.frontiersin.org/about/PublishingFees

